\documentclass{article}
\usepackage{listings}
\usepackage{hyperref}
\usepackage{url}
\usepackage[letterpaper, hmargin=1.25in, vmargin=1.5in]{geometry}

\title{Adding a new model to NPSGD}
\author{Thomas Dimson}

\setlength{\parskip}{0.5\baselineskip}
\newcommand{\mpath}[1]{\texttt{#1}}
\newcommand{\mclass}[1]{\texttt{#1}}
\newcommand{\mmethod}[1]{\texttt{#1}}
\lstset{ %
    language=Python,                % choose the language of the code
    showspaces=false,               % show spaces adding particular underscores
    showstringspaces=false,         % underline spaces within strings
    showtabs=false,                 % show tabs within strings adding particular underscores
    breaklines=true,                % sets automatic line breaking
    breakatwhitespace=false,        % sets if automatic breaks should only happen at whitespace
    basicstyle=\ttfamily
}

\begin{document}
\maketitle
\tableofcontents
\newpage

\section{Introduction}
NPSGD was created with the design goal of making the addition of new models
easy. All models are actually just python classes that inherit from a specific
base class. By adding names and specific parameters, NPSGD will do the work of
shipping the parameters from the web interface into the module. It is up to the
module to perform the work of running the model and giving meaningful output to
the user.

\section{Quick Start}
All models are added to the \mpath{models/} subdirectory and are really just python
modules in disguise. The quickest start possible is to copy \mpath{example.py} to a
new filename called \mpath{my\_model.py} and change \mpath{short\_name} to
\mpath{my\_model}. 
Visit \url{http://npsgdserver:8000/models/my_model} and your model will be up and running.

Please note that you will have to make sure that the model is available on all
servers running NPSGD daemons (queue, web, and workers).

\section{Models}
In NPSGD, a ``model'' is actually a Python class that inherits, however
distantly, from the base class of \mclass{ModelTask}. This gives the user the
flexibility to be language-agnostic in terms of model implementation, with a
quick Python wrapper as a model runner.

All models (with a .py extension) are placed in the \mpath{models/} subdirectory.
Periodically, the \textbf{queue}, \textbf{web},
and \textbf{worker} daemons will scan this directory for conforming (or newly
timestamped) classes and load them into memory. From there, the model will be
immediately accessible from the web, and therefore from the worker and queue. By
default, the scanner propagation time is 120 seconds.

\section{Helper classes}
\label{sec:helpers}
Rather than inheriting from \mclass{ModelTask} directly, NPSGD provides two
helper classes that can greatly speedup creation of models. These are
\mclass{MatlabTask} and \mclass{StandaloneTask}.

\subsection{StandaloneTask}
Models inheriting from \mclass{StandaloneTask} are wrappers for standalone executables
that communicate using command line arguments (e.g. specifying the path to
specific data input). When a model executes the pre-requisites are prepared
in the working directory, and a subprocess is spawned using the executable and
command line parameters. 

\subsection{MatlabTask}
\mclass{MatlabTask} aids in connecting models that were programmed in Matlab. Essentially, this
allows the user to access web parameters \textit{directly} in Matlab code,
bypassing the need worry about technical details such as spawning subprocesses
to run the model.

\section{NPSGD Pipeline}
The purpose of NPSGD is to perform all the administrative work needed to deliver
data to your model implementation. It is useful to understand exactly the
processes that occurs in order for this to happen, though the model implementor
will not have to touch this pipeline at all.
\begin{enumerate}
   \item Model, implemented in \mpath{models/example.py} is noticed and loaded by all NPSGD daemons 
   \item Online web user visits the web interface for the model, typically at
   \url{http://npsgdserver:8000/models/example}. The user specifies all the parameters
   present in the model and clicks submit
   \item Web daemon \textbf{npsgd\_web} submits the request to the queue
   \item Queue daemon, \textbf{npsgd\_queue} sends an email to the user with a
   confirmation code for the request.
   \item User receives email, clicks the confirmation code link which is
   typically \url{http://npsgdserver:8000/confirm_submission/code}. The web
   daemon submits the confirmation code to the queue daemon at this time.
   \item Queue daemon waits for a worker to poll. When a worker polls with this
   model available, queue will hand off the request. 
   \item Worker spawns the model class with the parameters that were specified
   in the web interface. Model proceeds with model pipeline documented in
   Section \ref{sec:ModelPipeline}
   \item Model pipeline completes, sends an e-mail to the user and then tells
   the queue that everything has completed successfully
\end{enumerate}

\section{Model Pipeline}
\label{sec:ModelPipeline}
The model pipeline is the section that a model implementor will actually have to
create in the model. Typically, these follow a strict order that is specified in
\mclass{ModelTask} class under the method \mmethod{run}:
\begin{enumerate}
    \item A working directory is created for the model (usually
    \mpath{/var/tmp/npsgd/unique\_id}. 
    \item Execution is setup in the \mmethod{prepareExecution} method. This is
    generally done by converting parameters into a form (e.g. a file in the
    working directory) that the model can work.
    \item The model is run under the \mmethod{runModel} method. A typical model
    run includes calling the subprocess with specified parameters and waiting
    for the result.
    \item Graphs are prepared under the \mmethod{prepareGraphs} method using the
    data that was outputted from runModel.
    \item PDF document is created in \mmethod{generatePDF}, through
    \mmethod{getAttachments} using PDFLatex. 
    \item E-mail is sent using \mmethod{sendResultsEmail}
\end{enumerate}

Note that the pipeline can be greatly simplified by using the helper methods
outlined in Section \ref{sec:helpers}.

\section{Model Implementation}
This section is a basic guide about specifying a model to run.

\subsection{Defining a class}
In order to create a model, you must inherit from \mclass{ModelTask}, or one of
the helpers specified in Section \ref{sec:helpers}. After doing so, you have to
save your model under \mpath{models/my\_model.py}. For example, a skeleton
model could take the form of
\begin{lstlisting}
    from npsgd.model_task import ModelTask
    from npsgd.model_parameters import *

    class MyModel(ModelTask):
        short_name = "my_model"
        full_name  = "My Model"
        subtitle   = "The best model money can buy"
\end{lstlisting}

\subsection{Parameters}
Models are pretty useless without parameters which the user can play around with
in order to get different result. All parameters are specified in
the class structure itself, under a list with the name \mpath{parameters}. As an
example to start, we could specify an integer representing the number of samples
to take into account during a simulation, with a default of 1000:
\begin{lstlisting}
    ...
    class MyModel(ModelTask):
        ...
        parameters = [
            ...
            IntegerParameter('nSamples', description="Number of samples to specify",
                default=1000)
            ...
        ]

\end{lstlisting}

\subsubsection{Parameter Options}
All parameters support a variety of different options in order to modify
behaviour. The first value passed into the parameter is always the
\texttt{name}, which is used as a reference through NPSGD. The rest of the
options are specified using keyword arguments. Each parameter type outlined in
Section \ref{sec:ParameterTypes} has a superset of these options:\\
\begin{tabular}{ r p{4in} }
    \textbf{Option} & \textbf{Description} \\
    \hline
    name & Unique identifier used to reference the parameter through NPSGD.
           These generally should not have spaces, and by convention are camel
           cased. \\
    description & As it says, used to describe the parameter. The string
                  specified here will appear in a number of places, including
                  the output \LaTeX document, and the html page. \\
    default & Used to specify the default value of a parameter. \\
    hidden  & Boolean used to toggle whether the parameter is hidden or not.
              Currently, this only affects the output html. Hidden parameters will simply
              be passed along with their default value. This is useful for
              subclassing models\\
    helpText & Used to give ``helpful hints'' when a user is specifying the
               model. This currently appears as balloon text on the html output
               page.
\end{tabular}

\subsubsection{Parameter Types}
\label{sec:ParameterTypes}
Currently, NPSGD supports the following parameter types:
\begin{itemize}
    \item \texttt{StringParameter}: Basic string input. The only extra option is
              \texttt{units}, which represents the units of input.

    \item \texttt{FloatParameter}: Basic float input. Float parameters can be
              used to specify a specific range of inputs by specifying the
              \texttt{rangeStart}, \texttt{rangeEnd} and \texttt{step} inputs. If both
              \texttt{rangeStart} and \texttt{rangeEnd} are specified, the float
              parameter will function as a slider. If only \texttt{rangeStart} or
              \texttt{rangeEnd} are specified, the input will be clamped to values
              accordingly. Additionally, floats take a \texttt{units} parameter that
              can be used to specify the units of the float (e.g. nm, cm)

    \item \texttt{IntegerParameter}: Basic integer input. Integers have exactly
              the same options as \texttt{FloatParameter}, but will be verified to
              ensure integrality.

    \item \texttt{RangeParameter}: Range parameters are used for specifying a
              range of floating point inputs (e.g. 400-2500nm). Options for
              \texttt{RangeParameter} match \texttt{FloatParameter}, but must
              have all of \texttt{rangeEnd}, \texttt{rangeEnd} and \texttt{step}
              specified. The output and default values of a range must be
              specified as a pair of floats, e.g. \texttt{default=(1,5)},
              representing the range of choices.

    \item \texttt{SelectParameter}: Select parameters are used to clamp input to
              a specific set of options, something like a combo box. Select
              parameters have an option called \texttt{options}, which is a list
              of valid inputs that the select box can take (e.g. ``Strong'',
              ``Weak'').

    \item \texttt{BooleanParameter}: Boolean parameters are similar to the
              \texttt{SelectParamater} type, but now are clamped to true/false. 
              These parameters will display as text boxes in html and have no
              additional options.
\end{itemize}

\subsubsection{Attachments}
\label{sec:attachments}
By default, NPSGD includes only the PDF created via \LaTeX, outlined in Section
\ref{sec:LatexOutput} as an attachment. If your model creates more output (such
as data files, graphs, pictures, etc.) then you may want to add additional
attachments. In NPSGD, this is performed by adding a class variable by the name
of \texttt{attachments} consisting of a list of all additional attachments
within the working directory to include. For example,
\begin{lstlisting}
    ...
    class MyModel(ModelTask):
        ...
        attachments = [picture.jpg, data.txt]
        ...
\end{lstlisting}
The above code listing would include two e-mail attachments (\mpath{picture.jpg}
and \mpath{data.txt}) along with the usual \mpath{results.pdf}.

If you want more flexibility in specifying attachments, consider overriding the
\texttt{getAttachments} method on \texttt{ModelTask}.

\subsubsection{Graphical Output}
After an executable has run, usually a task will create graphs out of the output
In NPSGD, this can be accomplished by saving files in the working
directory by overriding the \texttt{prepareGraphs} method in a model. Python has
an excellent library called \texttt{matplotlib}
(\url{http://matplotlib.sourceforge.net/}) which creates graphs that are on-par
with Matlab's using commands that are almost identical to Matlab plot syntax.
For example,
\begin{lstlisting}
    ...
    import os
    import matplotlib
    matplotlib.use("Agg") #suppress graphical user interface
    import matplotlib.pyplot as plt
    ...
    class MyModel(ModelTask):
        ...
        attachments = [plot.png]
        ...
        def prepareGraphs(self):
            x = [1,2,3,4,5]
            y = [10,5,2,6,7]
            plt.clf() #clear previous plot
            plt.plot(x,y)
            plt.xlabel("X")
            plt.ylabel("Y")
            plt.title("Demo Plot")
            plt.savefig(os.path.join(self.workingDirectory, "plot.png"))
        ...
\end{lstlisting}

Note that the \texttt{prepareGraphs} method is completely optional - if your
model does not have graphical output, or creates the output inside the
executable then you may include the relevant files using the attachment mechanism
outlined in Section \ref{sec:attachments}.


\subsubsection{Latex Output}
\label{sec:LatexOutput}
Output for all models is generally routed through PDFLatex. Each model will
\textbf{definitely} want to override the \texttt{latexBody} method. This method
returns a string that is then run through PDFLatex in order to generate a file
called \mpath{result.pdf}. This file, by default, is included in every e-mail
that is sent out.

\mclass{ModelTask} has a useful method for creating a \LaTeX table containing
all the parameters that the user has specified. By including
\texttt{self.latexParameterTable()} somewhere in the \LaTeX output the output
pdf will contain a very nicely formatted parameter table.

A complete example:
\begin{lstlisting}
    ...
    class MyModel(ModelTask):
        ...
        def latexBody(self):
            return r"""
            Hello there!

            \section{Main Section}
            This is the result of an example model run for NPSGD. Your
            parameters were:

            %s
""" % self.latexParameterTable()
\end{lstlisting}

It is highly recommended that you use python docstrings (triple quoted strings)
in order to specify output, as well as using the \texttt{r} prefix to the string
(raw string mode, so you don't have to escape slashes).

\subsubsection{Reading Parameter Values}
When executing your script, preparing graphs and outputting \LaTeX it is often
necessary to have access to the parameters that the user has specified at the
web interface. These are \textit{automatically} delivered to the script using
the names that you specified in for your parameters. By accessing
\texttt{self.parametername.value} in any instance method, you will get access to
the value that the user specifies. This is best illustrated by example:

\begin{lstlisting}
    ...
    class MyModel(ModelTask):
        ...
        parameters = [
            ...
            IntegerParameter('nSamples', description="Number of samples to specify",
                default=1000)
            ...
        ]

        def prepareGraphs(self):
            print self.nSamples.value #Will output the number of samples the
                                      #user specified

\end{lstlisting}

\subsubsection{Helpers: StandaloneTask}
Subclassing \mclass{StandaloneTask} automates the process of running a
subprocess in order to execute a model on the command line. This is the most
technical part of the process.

\mclass{StandaloneTask} specifies a method of \texttt{runModel} that simply
executes a command as a python subprocess, and stores the stderr/stdout of the
subprocess in instance parameters \texttt{self.stdout} and \texttt{self.stderr}.
The subprocess is executed within the model's working directory. 

The model creator must specify one additional parameter, and one additional
method for running. The class variable \texttt{executable} specifies the path to
the executable we wish to run (typically the model executable, or something like
java for a java task). The instance method \texttt{executableParameters} returns
a list of parameters for the executable along with values. The parameters are
specified as a python list. This is best shown via example:

\begin{lstlisting}
    from npsgd.model_task import StandaloneTask
    from npsgd.model_parameters import *

    class LsModel(StandaloneTask):
        ...
        executable = "ls"
        ...
        def executableParameters(self):
            return [
                "-al",
                "/var/tmp"
            ]
\end{lstlisting}

Such a model will execute \texttt{ls} in a subprocess and return the results in
\texttt{self.stdout}. Many more examples ship along with NPSGD.

\subsubsection{Helpers: MatlabTask}
Subclassing \mclass{MatlabTask} automates the process of spawning a matlab
subprocess, which can be a tricky and time consuming process. It also delivers
the parameter values \textbf{directly} into the matlab environment. The script
will just ``magically'' have access to all the values of user input in variables
that match the names specified in the model class. 

A user of the \mclass{MatlabTask} helper need only specify one additional class
parameter, namely \texttt{matlabScript} which gives the location of the script
Matlab should execute. A full example of using \mclass{MatlabTask} is specified
in Appendix \ref{sec:MatlabExample}.

\appendix

\section{Complete Example of a Matlab Task}
\label{sec:MatlabExample}
\subsection{NPSGD Code}
\lstinputlisting{../models/example.py}

\subsection{Matlab Code}
\lstinputlisting[language=Matlab]{../models/example/example.m}

\section{Complete Example of ABM-U}
ABM-U is a more sophisticated example of a model task. An example of this code
running is available at \url{http://www.npsg.uwaterloo.ca/models/ABMU.php}.
\lstinputlisting{../models/abmu_c.py}

\end{document}
