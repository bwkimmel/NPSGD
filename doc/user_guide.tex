\documentclass{article}
\usepackage{listings}
\usepackage{hyperref}
\usepackage{url}
\usepackage[letterpaper, hmargin=1.25in, vmargin=1.5in]{geometry}

\title{NPSGD User Guide}
\author{Thomas Dimson}

\setlength{\parskip}{0.5\baselineskip}
\newcommand{\mpath}[1]{\texttt{#1}}
\newcommand{\mclass}[1]{\texttt{#1}}
\newcommand{\mmethod}[1]{\texttt{#1}}
\lstset{ %
    language=Python,                % choose the language of the code
    showspaces=false,               % show spaces adding particular underscores
    showstringspaces=false,         % underline spaces within strings
    showtabs=false,                 % show tabs within strings adding particular underscores
    breaklines=true,                % sets automatic line breaking
    breakatwhitespace=false,        % sets if automatic breaks should only happen at whitespace
    basicstyle=\ttfamily
}

\begin{document}
\maketitle
\begin{abstract}
    NPSGD is an online framework that makes it easy for scientists to expose
    their models online. This document provides an general overview of how the
    system works with some administrative tips along the way. Notably, it does
    not talk about the inner workings of how to create models as this is
    documented in the guide called ``Adding a new model to NPSGD''.
\end{abstract}
\tableofcontents

\section{System Requirements}
A basic configuration will require:
\begin{itemize}
    \item Python 2.5 or higher \url{http://www.python.org/}
    \item Facebook's Tornado Web server for Python
    \url{http://www.tornadoweb.org/}
    \item \LaTeX distribution of some form.
    \item If you are going to use some Matlab scripts, some version of Matlab is
    required
    \item UNIX-like operating system. Cygwin may work but has not been tested.
\end{itemize}


\section{System Overview}
\subsection{Daemons}
NPSGD is a distributed system which communicates internally and externally via
http. One of the design goals of the system was to allow multiple machines to
perform the function of worker so as not to overload any particular system. To
fascilitate this, the framework is split into three daemons:
\begin{itemize}
    \item \texttt{npsgd\_web}: The frontend to NPSGD. The job of this daemon is
    to communicate with all external clients over http. Since no state is kept
    inside the daemon, you can have as many frontends as desired (though one
    should be sufficient unless you are under heavy load). By default, a
    client visiting \url{http://localhost:8000} will hit this daemon.

    \item \texttt{npsgd\_queue}: The manager of NPSGD. The job of this daemon is
    to keep track of the state of particular model runs, send out e-mail
    confirmations, and to distribute jobs to worker machines. There can only
    ever be one queue within an NPSGD system. By default, it runs on port 9000
    but should not be accessed externally (in particular, it should be
    firewalled off).

    \item \texttt{npsgd\_worker}: The model processor of NPSGD. The job of these
    daemons are to poll the queue for model requests and process them.
    Generally, this takes the form of a finite state machine that spawns some
    external process to actually perform the scientific simulation. After
    processing, it sends results to the requestor's e-mail address.
\end{itemize}

If NPSGD is being configured for long term access, all of these daemons should
have startup scripts associated on the end machine. Sample scripts are available
in the \mpath{startup/} directory alongside the NPSGD distribution.

Each model that is implemented for NPSGD must be available for the queue, the
web frontend and the worker machines. A NFS mount for the models is particularly
handy for this purpose (though not necessary).

\subsection{Request Flow}
A sample model of NPSGD is available at
\url{http://www.npsg.uwaterloo.ca/models/ABMU.php}. It would be helpful to try
this out in order to gain understanding of the system. The typical request flow
would be something like this:
\begin{enumerate}
    \item User visits the model site via \url{http://localhost:8000/models/example},
    which is served using \texttt{npsgd\_web}.

    \item User submits a request for the example model, \texttt{npsgd\_web} makes
    a request to \texttt{npsgd\_queue} to submit a confirmation code.

    \item \texttt{npsgd\_queue} e-mails a confirmation code to the user's e-mail
    address.
    
    \item User clicks the link in the e-mail, usually something like
    \url{http://localhost:8000/confirm\_submission/code}. 

    \item \texttt{npsgd\_web} forwards the confirmation onto
    \texttt{npsgd\_queue}. At this point the model request is ready to be
    processed.

    \item \texttt{npsgd\_worker} polls the queue for a task. Since there is a
    task, it receives a task response.

    \item \texttt{npsgd\_worker} prepares execution, spawns a subprocess to open
    up the model run. 
    
    \item Model execution completes. \texttt{npsgd\_worker} creates a \LaTeX
    document with the results.

    \item \LaTeX document and other attachments are sent to the user's e-mail.
    \texttt{npsgd\_worker} tells the queue that it has executed successfully.
\end{enumerate}

\subsection{Fault Tolerance}
NPSGD has been designed to be fault tolerant. Whenever a web request is made,
\texttt{npsgd\_web} will check with the queue to ensure it is up and that it has
workers on the line that have polled for a request recently. If one of these
conditions is not met, a simple error message is displayed to the requestor.

When processing a task, \texttt{npsgd\_workers} communicate back to the queue
with a heartbeat request to make sure that the queue knows that the worker is
still alive. If this heartbeat has not been heard for a long while, the queue
daemon will put the request back into the queue.

All requests are tried for a fixed number of times before a complete failure (by
default, the queue will retry a request three times before declaring it failed).
Upon failure, an error message is e-mailed to the user.

If the e-mail system is ever down, NPSGD will keep trying to send e-mails out
until they have successfully completed.

One particular ``gotcha'' at this point is that the queue server is not
persistant: all requests are kept in memory. This makes it particularly
important to make sure it does not go down as requests could be lost if it does
it midway through processing. A persistant queue file is a large possibility for
future development.

\section{Software Layout}
NPSGD ships in exactly the same way it was developed and isn't easily converted
into a form that matches the standard UNIX layouts. When you clone the main
repository for NPSGD, you will see the following files and directories:
\begin{itemize}
    \item \mpath{config.example}: An example configuration file for NPSGD
    \item \mpath{epydoc.cfg}: Config file for generating an HTML representation
          of the code using epydoc.
    \item \mpath{LICENSE}: License agreement for distribution
    \item \mpath{README}: Easy readme guide to NPSGD
    \item \mpath{npsgd\_web.py}: Web daemon for handling client requests.
    \item \mpath{npsgd\_queue.py}: Daemon for queueing NPSGD model runs.
    \item \mpath{npsgd\_worker.py}: Worker for actually processing NPSGD model
          runs.
    \item \mpath{doc/}: Directory containing NPSGD documentation.
    \item \mpath{models/}: Default directory for storing user-defined models
    \item \mpath{npsgd/}: Helper package containing modules required to run the
          daemons.
    \item \mpath{startup/}: Directory containing some sample startup scripts for
          Ubuntu (must be modified to specify paths correctly).
    \item \mpath{static/}: Directory containing static files served by the web
          server (Javascript, css).
    \item \mpath{templates/}: Directory containing templates used by NPSGD
\end{itemize}


\section{Development Environment}
Since NPSGD consists of so many moving parts, creating a quick development
environment can be quite a task. First of all, ensure that all the system
requirements are installed. After that, you will want to clone NPSGD from the
main development site and then make a custom copy of the config file.
\begin{verbatim}
git clone git://github.com/cosbynator/NPSGD.git
cp config.example config.cfg
\end{verbatim}

You will need to edit the config file to make a few changes, most notably:
\begin{itemize}
    \item Change \texttt{npsgbase} to point at your development directory
    \item Change \texttt{pdflatexPath} to point at your \texttt{pdflatex}
    binary.
    \item Change the \texttt{email} section to specify a valid
    username/password. The defaults are approximately set up for a gmail
    account (obviously with an invalid username/password).
    \item Change \texttt{matlabPath} if you will be testing/editing matlab
    models
\end{itemize}

After completing these sections you will be able to run all three daemons on a
local terminal. Specifically, run the following in separate terminal windows:
\begin{verbatim}
python npsgd_web.py
python npsgd_queue.py
python npsgd_worker.py
\end{verbatim}
Logging will be printed to standard error unless otherwise specified. After
running \texttt{npsgd\_web} you should be able to browse to
\url{http://localhost:8000/models/example} and see some sample output. Try
performing a complete model run to make sure the system is operating correctly.

\section{A word on models}
Models are the key component to customizing NPSGD. The scope of this document
does not cover the \textit{implementation} of custom models (see ``Adding a new
model to NPSGD'' for that). Still, it is important to understand how they fit
into the system.

Models are really just Python files that inherit from \mclass{ModelTask} in
order to provide complete flexibility for underlying implementation. These
python files often act as wrappers to existing models that are implemented in
some other language (we have models implemented in both Matlab and C++). Just
because they are Python files does not force the implementation to be in Python:
Python just acts as a facilitator between NPSGD and the underyling model. 

Each daemon is configured to monitor the model directory, by default under
\mpath{models/}. Python files in this directory act as ``plugins'' to the system
- models are implemented in them. The daemons periodically scan the directory
and look at the all the python files, those that inherit from \mclass{ModelTask}
are imported into the system. When importing, a hash of the file is used to tag
the model with a ``version''. Duplicate versions of models are ignored (not
imported into the system), while new versions are stored. Old versions are
retained in the system in case there are any queued requests that are still
acting on the old version. For this reason, it is important to reload the page
after updating any models.

Models are self-contained: they have all the information necessary for the
queue, web and worker daemons to operate. In particular, they contain the
parameters necessary for the web interface to configure the models and the
execution cycle necessary in order for the workers to run a model with a
particular parameter set. The code for the models must be shared
across the daemons: this is done manually, or via a NFS mount.

\section{Templates}
Templates have many purposes in NPSGD:
\begin{enumerate}
    \item To create the HTML that is served by the web daemon
    \item To create the e-mail sent by the queue and worker daemons
    \item To create \LaTeX markup to send results to the user
\end{enumerate}

All templates for npsgd are stored in the \mpath{templates/} subdirectory. This
is meant to be user defined. The default templates shipped with NPGSD are quite
specific to our own models.

Templates use a syntax created for the Tornado web server. This syntax is
documented at the Tornado site: \url{http://www.tornadoweb.org/}, under
templates.

\subsection{HTML Templates}
NPSGD ships with two sets of html templates, one for an embedded site (served
directly without things like title tags), and one for a basic site. These
templates are used by the web daemon to display html directly to the user. Both are
available under the \mpath{templates/html/} subdirectory. These files contain:
\begin{enumerate}
    \item \texttt{base.html}: Template that all other templates inherit from.
    Could contain things like HTML headers.

    \item \texttt{model.html}: The template that displays a model to the user.

    \item \texttt{model\_error.html}: The template that shows an error when a
    something goes wrong in the model.

    \item \texttt{confirm.html}: Template that shows after a model request has
    been made (displays that an e-mail will be dispatched to the user).

    \item \texttt{confirmed.html}: Template that shows after a confirmation code
    has been entered correctly.

    \item \texttt{already\_confirmed.html}: Template that shows after a
    confirmation code has been entered for a second time. This displays a
    message saying that the model will not be rerun.
\end{enumerate}

\subsection{Email Templates}
Emails are used as communication to the user after a request has been performed
via HTML. Each template comes in two parts: one for the email subject, and one
for the email body. The naming convention is ``name\_body.txt'' for body and
``name\_subject.txt'' for subject. In particular, these templates are available:
\begin{enumerate}
    \item \texttt{results\_email}: The email used to display model run results
    \item \texttt{failure\_email}: The email used to notify the user that we
    could not complete a model request.
    \item \texttt{confirm\_email}: The email used to communicate the
    confirmation code for a particular model request.
\end{enumerate}

\subsection{LaTeX Templates}
There is only one LaTeX template, namely \mpath{result\_template.text}. This is
used to declare all the packages required for a model results PDF. Inside, there
is an empty variable where the details of a model run will go. Models themselves
specify the model details, but it is always wrapped in this template.

\section{Static Files}
In addition to templates, NPSGD needs the use of certain static files to
customize the in-browser behaviour. All static files are available in the
\mpath{static/} subdirectory.

\subsection{Javascript}
NPSGD uses a lot of javascript to do paramater verification and provide UI
widgets for the user. In the \mpath{static/js} subdirectory, you will find
\begin{enumerate}
    \item \texttt{npsgd.js}: our particular javascript files
    \item \texttt{jquery-version.min.js}: JQuery, a javascript library that
    makes writing javascript simpler: \url{http://jquery.com/}
    \item \texttt{jquery.qtip.min.js}: A library for creating popup tooltips.
    This is used for model's helper text.
    \url{http://craigsworks.com/projects/qtip/}
    \item \texttt{jquery-ui-version.min.js}: JQuery UI, a set of UI widgets for
    JQuery: \url{http://jqueryui.com/}. This is used to give a nice user
    experience for things like sliders and range selectors.
    \item \texttt{jquery.validate.min.js}: JQuery validation plugin. We use this
    to perform parameter verification in client side, before requests are sent
    to the server:
    \url{http://bassistance.de/jquery-plugins/jquery-plugin-validation/}
\end{enumerate}

\subsection{CSS}
In addition to Javascript, some basic CSS needs to be shared across all
templates. These are in the \mpath{static/css} subdirectory.
\begin{enumerate}
    \item \texttt{npsgd.css}: This provides just a couple of CSS includes that
    are needed across all templates.
    \item \texttt{smoothness/}: This subdiretory contains the default CSS files
    for the JQuery UI project. It is used to dislay the widgets of JQuery UI.
\end{enumerate}

\subsection{Images}
Finally, there are some images that NPSGD needs to display available in the
\mpath{static/images} subdirectory. These should mostly be self explanitory. 

\end{document}
